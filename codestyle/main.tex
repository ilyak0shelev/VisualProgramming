\documentclass{article}
\usepackage[utf8]{inputenc}
\usepackage[english,russian]{babel}
\usepackage{xcolor}
\usepackage{color}
\usepackage{listings}
\usepackage{caption}

\title{Code Style}
\author{Кошелев И.А. гр. ИА-032}
\date{February 2022}

\begin{document}
\maketitle

\section{Введение}
В данном документе описан мой стиль написания кода.

\section{Основная часть}
\lstset{ %
language=C++,                 % выбор языка для подсветки (здесь это С++)
basicstyle=\small\ttfamily, % размер и начертание шрифта для подсветки кода
numbers=left,               % где поставить нумерацию строк (слева\справа)
numberstyle=\tiny,           % размер шрифта для номеров строк
stepnumber=1,                   % размер шага между двумя номерами строк
numbersep=5pt,                % как далеко отстоят номера строк от подсвечиваемого кода
backgroundcolor=\color{lightgray}, % цвет фона подсветки - используем \usepackage{color}
showspaces=false,            % показывать или нет пробелы специальными отступами
showstringspaces=false,      % показывать или нет пробелы в строках
showtabs=false,             % показывать или нет табуляцию в строках
frame=false,              % рисовать рамку вокруг кода
tabsize=2,                 % размер табуляции по умолчанию равен 2 пробелам
captionpos=t,              % позиция заголовка вверху [t] или внизу [b] 
breaklines=true,           % автоматически переносить строки (да\нет)
breakatwhitespace=false, % переносить строки только если есть пробел
escapeinside={\%*}{*)}   % если нужно добавить комментарии в коде
}

\lstset{language=С++}
\textbf{\emph{Используемые языки программирования}}
\begin{itemize}
    \item C/C++;
\end{itemize}
\textbf{\emph{Пробелы и отступы}}
\\Операторы и операнды разделяются пробелом:
\begin{lstlisting}
int x = (a + b) * c / d;
\end{lstlisting}
Также пробелом отделяются фигурные скобки:
\begin{lstlisting}
if (a > 10) { return }
\end{lstlisting}
\textbf{\emph{Оформление циклов и операторов управления}}
\\При использованиии циклов и операторов управления всегда используются фигурные скобки, отступы и переносы на новую строку, там где это нужно:
\begin{lstlisting}
if (temp == 10) {
    return temp;
} else {
    for (i = 0; i < 10; i++) {
        std::cout << "error" << std::endl;
    }
}
\end{lstlisting}
Цикл for используется, когда известно количество повторений, а цикл while, когда
количество повторений неизвестно:
\begin{lstlisting}
for (i = 0; i < 10; i++) {
    ...
}

while (i < 10) {
    ...
}
\end{lstlisting}
\textbf{\emph{\\Разделение функций и блоков кода}}
\\Функции и различные по смыслу блоки кода разделяются пустой линией:
\begin{lstlisting}
int smthFunc1 () {
}

int smthFunc2 () {
}
\end{lstlisting}
\textbf{\emph{Названия функций и переменных}}
\\Функции и переменные должны иметь описательные и логичные названия, не приветствуются однобуквенные названия (если это не итераторы).
\textit{\\Названия переменных:}
\begin{lstlisting}
//counter variable:
int m = 0; // wrong name
int counter = 0; // correct name
\end{lstlisting}
\\Названия методов и функций должны быть записаны в смешанном регистре и начинаться с нижнего.
\textit{\\Названия методов и функций}
\begin{lstlisting}
getName();
addFirst();
\end{lstlisting}
\\Именованные константы должны быть записаны в верхнем регистре с нижним подчёркиванием в качестве разделителя.
\textit{\\Названия констант}
\begin{lstlisting}
MAX_ITERATIONS, COLOR_RED
\end{lstlisting}
\textbf{\emph{Оформление классов}}
\\Класс следует объявлять в заголовочном файле и определять (реализовывать) в файле исходного кода, имена файлов совпадают с именем класса:
\begin{lstlisting}
MyClass.h, MyClass.cpp
\end{lstlisting}
\\Заголовочные файлы объявляют интерфейс, файлы исходного кода его реализовывают.
\\Не следует объявлять переменные класса как public. Вместо этого нужно использовать переменные с модификатором private и соответствующие функции доступа:
\begin{lstlisting}
class MyClass
{
public:
  int getValue () {
    return value_;
  }
  ...

private:
  int value_;
}
\end{lstlisting}
\section{Вывод}
В ходе работы я описал свой стиль создания кода, написал отчёт в редакторе \LaTeX{}.
\begin{thebibliography}{3}
 \bibitem{1} Осваиваем LaTeX за 30 минут [Электронный ресурс] URL: https://habr.com/ru/company/ruvds/blog/574352/ (дата обращения 06.02.2022)
 \bibitem{2} C++ Programming Style Guidelines [Электронный ресурс] URL: https://geosoft.no/development/cppstyle.html (дата обращения 06.02.2022)
 \bibitem{3} Оформление исходного кода в документах LaTeX [Электронный ресурс] URL: http://dkhramov.dp.ua/Comp.CyrillicInListingsTex#.Ygk7599ByUk (дата обращения 06.02.2022)
\end{thebibliography}
\end{document}